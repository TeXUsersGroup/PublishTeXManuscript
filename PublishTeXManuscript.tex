\documentclass{article}
\usepackage{natbib}
\usepackage[colorlinks]{hyperref}
\begin{document}
\title{So, you decided to publish your \TeX\ manuscript?  How to talk
  to the publisher}
\author{Boris Veytsman\thanks{Copyright 2016, \TeX\ Users Group,
    available under Creative Commons Attribution License (CC-BY)}}
\date{December 2016}
\maketitle

\section{Introduction}
\label{sec:intro}

Some time ago Dominik Wujastyk asked an interesting question: how do
we talk to a (possibly) \TeX-ignorant publisher if we want her or him
to accept a manuscript in \TeX?  A horror story~\cite{Verna13} shows
that this is an important problem.

This lead to the creation of this document.  Here we try to describe
the common problems with \TeX-averse publishers, arguments you can
use, etc.


Please feel free to add your own advice (and your name to the authors
list!) to this file.  

While many people user \LaTeX\ for document preparation, our
discussion is not limited to it: we would be glad to hear from Con\TeX
t people and the users of other formats.  


\section{Manuscript preparation}
\label{sec:preparation}

Before you even started to discuss your manuscript with a publisher,
try looking at it from the publisher's point of view.  For many
publishers \TeX\ manuscripts have the reputation of being very
difficult to work with.  One of the main reasons is the versatility of
\TeX, and the abuse of this versatility by the authors.  Many authors
tend to define their own macros, and some just copy and paste page
after page of definitions from their own papers, colleagues' paper and
every \TeX-related Web page on the Internet.  Even if the publisher
uses \TeX\ for her own back office production (more on this in
Section~\ref{sec:backoffice}), the manuscripts of this style are
almost not usable for her: the macros may conflict with the
publisher's macros or make it impossible to put the manuscript in the
house style.  The same with authors' packages: they may add too many
difficulties to the overworked technical editors.  In some cases the
publisher is forced to throw \TeX\ file away and completely re-key the
manuscript from the PDF---which is very expensive.

Thus if you want your publisher to love your \TeX\ manuscript:
\begin{enumerate}
\item Find out whether the publisher has ``official'' \TeX\ style.  If
  yes, use it.
\item Beyond this style, try not to use any additional packages or fonts.
\item Convert your own macros to their meaning: sometimes you need a
  smart editor program to do this.  \marginpar{TODO: List of such
    programs}. 
\end{enumerate}

You should also prepare a version for reviewers: with generous spacing
and margins for their comments.

\section{Advantages of a \TeX-formatted manuscript}
\label{sec:advantages}

In this section we list some advantages of a manuscript formatted with
\TeX\ from the point of view of a publisher.

\subsection{Integrity}
\label{sec:integrity}

A manuscript formatted in a word processor may look differently in
different versions of the word processor, or on different platforms.
This is especially true for such objects as equations, special symbols
etc.  While Unicode solves \emph{some} of these problems, still it is
quite common to see a word processor file with garbled equations or
text.

A manuscript in \TeX\ is guaranteed to be the same on the author's
computer and on the publisher's or reviewer's one.

\subsection{Postprocessing}
\label{sec:postprocessing}

Being a structured text format, \TeX\ allows easy postprocessing
options for metadata extraction, XML translation, archiving etc.  The
representative of the publisher may not even know that her processing
division would be happy to accept \TeX\ files.  Do some research about
the methods used by the people there.  

\subsection{Backoffice typesetting}
\label{sec:backoffice}

Many publishers outsource the actual typesetting to companies, that
use \TeX\ internally.  They convert word processor manuscripts to
\TeX\ (either using automatic tools or re-keying) and typeset them in
\TeX.  A manuscript already in \TeX\ might save time and effort, and
introduce less errors.

\section{Some publishers' arguments}
\label{sec:arguments}

In this section we discuss some typical publishers' arguments against
\TeX\ and your possible answers.


\subsection{My reviewers prefer word processors}
\label{sec:reviewers}

Actually many peer reviewers use \TeX\ themselves and would prefer a
\TeX\ file.  Even those who do not would be glad to have a nicely
typeset PDF file suitable for review (see
Section~\ref{sec:preparation}).  

\subsection{My copy editors prefer word processors}
\label{sec:copy-editors}

This is a very strong argument.  If your publishers' copy editors
refuse to take a manuscript in \TeX\ or PDF, there is little you can
do.  

You may try to convert your file to a word processor format, and then,
after copy editing, back to \TeX, but this is cumbersome and
error-prone.

Still, there are some possibilities.  In many cases your publisher's
representative just does not know actual copy editors employed by the
publisher, and is not aware that they \emph{can} work with \TeX\ or
PDF.  Doing your own research may give you some insight in the real
situation and help to convince the representative.

If your publisher outsources copy editing, you may suggest editors
willing to work with \TeX.  The \TeX\ consultants list maintained by
the \TeX\ Users Group at \url{http://www.tug.org/consultants.html} is
a good place to start.

\subsection{My typesetters prefer word processors}
\label{sec:typesetters}

In many cases this statement is just not correct (see also
Section~\ref{sec:backoffice}).  Again, you'd better come to the
publisher after doing you own research.

\section{\TeX-friendly publishers}
\label{sec:tex-aware-publishers}

In this section we list the publishers which accept manuscripts in
\TeX. \marginpar{\raggedright TODO: expand the list}
\begin{enumerate}
\item CRC Press, \url{https://www.crcpress.com/}.  
\item Elsevier, \url{https://www.elsevier.com/}
\item No Starch Press, \url{https://www.nostarch.com/}.
\item Springer, \url{http://www.springer.com/gp/}.
\end{enumerate}


\bibliographystyle{plainnat}
\bibliography{PublishTeXManuscript}

\end{document}
