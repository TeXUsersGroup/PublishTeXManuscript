\documentclass{article}
\usepackage{natbib}
\usepackage[colorlinks]{hyperref}
\begin{document}
\title{So, you decided to publish your \TeX\ manuscript?  How to talk
  to the publisher}
\author{Boris Veytsman\thanks{Copyright 2016, \TeX\ Users Group,
    available under Creative Commons Attribution License (CC-BY)}}
\date{December 2016}
\maketitle

\section{Introduction}
\label{sec:intro}

Some time ago Dominik Wujastyk asked an interesting question: how do
we talk to a (possibly) \TeX-ignorant publisher if we want her or him
to accept a manuscript in \TeX?  A horror story~\cite{Verna13} shows
that this is an important problem.

This lead to the creation of this document.  Here we try to describe
the common problems with \TeX-averse publishers, arguments you can
use, etc.


Please feel free to add your own advice (and your name to the authors
list!) to this file.  

While many people user \LaTeX\ for document preparation, our
discussion is not limited to it: we would be glad to hear from Con\TeX
t people and the users of other formats.  


\section{Manuscript preparation}
\label{sec:preparation}

Before you even started to discuss your manuscript with a publisher,
try looking at it from the publisher's point of view.  For many
publishers \TeX\ manuscripts have the reputation of being very
difficult to work with.  One of the main reasons is the versatility of
\TeX, and the abuse of this versatility by the authors.  Many authors
tend to define their own macros, and some just copy and paste page
after page of definitions from their own papers, colleagues' paper and
every \TeX-related Web page on the Internet.  Even if the publisher
uses \TeX\ for her own back office production (more on this later),
the manuscripts of this style are almost not usable for her: the
macros may conflict with the publisher's macros or make it impossible
to put the manuscript in the house style.  The same with authors'
packages: they may add too many difficulties to the overworked
technical editors.  In some cases the publisher is forced to throw
\TeX\ file away and completely rekey the manuscript from the
PDF---which is very expensive.

Thus if you want your publisher to love your \TeX\ manuscript:
\begin{enumerate}
\item Find out whether the publisher has ``official'' \TeX\ style.  If
  yes, use it.
\item Beyond this style, try not to use any additional packages or fonts.
\item Convert your own macros to their meaning: sometimes you need a
  smart editor program to do this.  \marginpar{TODO: List of such
    programs}. 
\end{enumerate}

\section{Some publishers' arguments}
\label{sec:arguments}

In this section we discuss some typical publishers' arguments and your
possible answers.  

\bibliographystyle{plainnat}
\bibliography{PublishTeXManuscript}

\end{document}
